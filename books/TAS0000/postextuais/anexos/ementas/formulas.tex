% formulas_condicionais.tex
\usepackage{ifthen} % Pacote necessário para condições
\usepackage{xparse} % Pacote para comandos avançados

% Exemplo 1: Fórmula com parâmetros simples
\newcommand{\formulaComParametro}[1]{
    \text{Resultado: } #1^2
}

% Exemplo 2: Fórmula com condições (ifthen)
\newcommand{\formulaCondicional}[1]{
    \ifthenelse{\equal{#1}{A}}{
        \text{Você escolheu a fórmula A: } E = mc^2
    }{
        \ifthenelse{\equal{#1}{B}}{
            \text{Você escolheu a fórmula B: } a^2 + b^2 = c^2
        }{
            \text{Escolha inválida. Use A ou B.}
        }
    }
}

% Exemplo 3: Usando xparse para mais flexibilidade
\NewDocumentCommand{\formulaAvancada}{ O{} m }{%
    \IfValueTF{#1}{
        \text{Parâmetro opcional: #1, Parâmetro obrigatório: #2}
    }{
        \text{Sem parâmetro opcional, Parâmetro obrigatório: #2}
    }
}
